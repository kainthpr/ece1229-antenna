%
% Copyright � 2015 Peeter Joot.  All Rights Reserved.
% Licenced as described in the file LICENSE under the root directory of this GIT rep{\textrm{LOS}}itory.
%
\makeoproblem{Corner cube antenna.}{advancedantenna:problemSet3:3}{2015 problem set 3, p3}{
\index{corner cube antenna}

Consider a symmetrically placed horizontal dipole antenna, next to a metallic corner cube.

\imageFigure{../figures/ece1229-antenna-figures/homework3Fig1}{A corner-cube antenna.}{fig:homework3:homework3Fig1}{0.2}

\makesubproblem{}{advancedantenna:problemSet3:3a}

Estimate the directivity enhancement of the antenna in \cref{fig:homework3:homework3Fig1} compared to the isolated antenna.

\makesubproblem{}{advancedantenna:problemSet3:3b}

Estimate the radiation resistance of the antenna in \cref{fig:homework3:homework3Fig1} compared to the isolated antenna.

\makesubproblem{}{advancedantenna:problemSet3:3c}

Calculate the array factor of the antenna in \cref{fig:homework3:homework3Fig1}.

\makesubproblem{}{advancedantenna:problemSet3:3d}

Plot the array-factor directivity pattern in the x-y plane for \( 0 < \phi \le 2 \pi \).

\makesubproblem{}{advancedantenna:problemSet3:3e}

By using numerical integration calculate the directivity of the array factor for \( h = (1/8) \lambda, h = (1/4) \lambda \) and \( h = (1/2) \lambda \).

} % makeoproblem

\makeanswer{advancedantenna:problemSet3:3}{

To compare intensity with and without the corner cube, it probably makes sense to set the origin at the location of the radiator in the corner-cube antenna.  Such a geometry
is sketched in 2D looking down, in
%\cref{fig:cornerCubeGeometry:cornerCubeGeometryFig2}.
\cref{fig:cornerCubeGeometry:cornerCubeGeometryFig3}.

%\imageFigure{../figures/ece1229-antenna-figures/cornerCubeGeometryFig2}{Corner cube geometry.}{fig:cornerCubeGeometry:cornerCubeGeometryFig2}{0.3}
\imageFigure{../figures/ece1229-antenna-figures/cornerCubeGeometryFig3}{Corner cube geometry.}{fig:cornerCubeGeometry:cornerCubeGeometryFig3}{0.3}

The primary source is located at \( \BS_0 = (0,0,0) \) and the image sources at \( \BS_\txtx = (-2h, 0, 0), \BS_y = (0, -2h, 0) \) behind the planes with the \( \xcap \) and \( \ycap \) normals respectively.  There are three wave vectors directions of interest, \( \kcap_{0} \) for the line of sight direction, \( \kcap_\txty \) for the reflection originating at the image source behind the z-x plane at \( -2 h \ycap \), and \( \kcap_\txtx \) for the reflection originating at the image source behind the y-z plane at \( -2 h \xcap \).  The points of reflection are \( \Bp_\txty \) and \( \Bp_\txtx \) respectively.

Suppose the magnetic vector potential has the structure of an infinitesimal dipole

\begin{equation}\label{eqn:advancedantennaProblemSet3Problem3:20}
\BA = \frac{\mu_0 I_0 }{4 \pi r} e^{-j k r} \zcap.
\end{equation}

In spherical coordinates \( \zcap = \cos\theta \rcap - \sin\theta \thetacap \), so the (far field) electric field at the observation point  \( \Bk = r \rcap = r ( \sin\theta \cos\phi, \sin\theta \sin\phi, \cos\theta) \) for the line of sight path is directed along

\begin{dmath}\label{eqn:advancedantennaProblemSet3Problem3:60}
\zcap - \lr{ \zcap \cdot \kcap} \kcap
=
\cos\theta \rcap - \sin\theta \thetacap - \cos\theta \rcap
=
- \sin\theta \thetacap.
\end{dmath}

The far field electric field is

\begin{dmath}\label{eqn:advancedantennaProblemSet3Problem3:80}
\BE
= -j \omega \BA_T
=  j \omega \frac{\mu_0 I_0 }{4 \pi r} e^{-j k r} \sin\theta \thetacap
=  j \eta k \frac{ I_0 }{4 \pi r} e^{-j k r} \sin\theta \thetacap.
\end{dmath}

In general the field will be dependent on the observation point inclination and azimuthal orientation.
For example, if the observation point is at \( \Bk = x \xcap \), the field will be

\begin{dmath}\label{eqn:advancedantennaProblemSet3Problem3:100}
\BE =  -j \eta k \frac{ I_0 }{4 \pi x} e^{-j k x} \zcap.
\end{dmath}

I didn't see a reason to expect to find a \( \sin\theta \) magnitude adjustment for the reflected image point sources, which would simplify the superposition calculation, but to zero-th order in \( h/r \) that is in fact the case.

To show this one must calculate the wave vector direction vectors to each of the reflection points.  After some algebra, those reflection points are found to be

\begin{subequations}
\label{eqn:advancedantennaProblemSet3Problem3:120}
\begin{dmath}\label{eqn:advancedantennaProblemSet3Problem3:140}
\Bp_x = \lr{ -h, \frac{h r \sin\theta \sin\phi}{r \sin\theta\cos\phi + 2 h}, \frac{h r \cos\theta}{r \sin\theta\cos\phi + 2 h}}
\end{dmath}
\begin{dmath}\label{eqn:advancedantennaProblemSet3Problem3:160}
\Bp_y = \lr{ \frac{h r \sin\theta \cos\phi}{r \sin\theta\sin\phi + 2 h}, -h, \frac{h r \cos\theta}{r \sin\theta\sin\phi + 2 h}},
\end{dmath}
\end{subequations}

%which have direction vectors proportional to
%
%\begin{subequations}
%\label{eqn:advancedantennaProblemSet3Problem3:180}
%\begin{dmath}\label{eqn:advancedantennaProblemSet3Problem3:200}
%\Bp_x \propto \lr{ - \sin\theta\cos\phi - 2 h/r, \sin\theta \sin\phi, \cos\theta}
%\end{dmath}
%\begin{dmath}\label{eqn:advancedantennaProblemSet3Problem3:220}
%\Bp_y \propto \lr{ \sin\theta \cos\phi, -\sin\theta\sin\phi - 2 h/r, \cos\theta}.
%\end{dmath}
%\end{subequations}

To zeroth order in \( h/r \), assuming that \( r \gg h \) the direction vectors towards \( \Bp_x \) and \( \Bp_y \) are

\begin{subequations}
\label{eqn:advancedantennaProblemSet3Problem3:240}
\begin{dmath}\label{eqn:advancedantennaProblemSet3Problem3:260}
\kcap_x = \lr{ - \sin\theta\cos\phi, \sin\theta \sin\phi, \cos\theta}
\end{dmath}
\begin{dmath}\label{eqn:advancedantennaProblemSet3Problem3:280}
\kcap_y = \lr{ \sin\theta \cos\phi, -\sin\theta\sin\phi, \cos\theta}.
\end{dmath}
\end{subequations}

Dotting these with \( \zcap \) to find the transverse projection in these wave vector directions

\begin{equation}\label{eqn:advancedantennaProblemSet3Problem3:300}
\kcap_x \cdot \zcap = \kcap_y \cdot \zcap = \cos\theta,
\end{equation}

so the far field electric fields towards the reflection points \( \Bp_x \) and \( \Bp_y \) are directed respectively along

\begin{subequations}
\label{eqn:advancedantennaProblemSet3Problem3:320}
\begin{dmath}\label{eqn:advancedantennaProblemSet3Problem3:340}
\BE_x
\propto (0,0,1) - \cos\theta \lr{ - \sin\theta\cos\phi, \sin\theta \sin\phi, \cos\theta}
=
\lr{ \sin\theta\cos\theta\cos\phi, -\sin\theta\cos\theta \sin\phi, \sin^2\theta}
=
-\sin\theta
\lr{ \cos\theta\cos\phi, -\cos\theta \sin\phi, -\sin\theta},
\end{dmath}
\begin{dmath}\label{eqn:advancedantennaProblemSet3Problem3:360}
\BE_y
\propto (0,0,1) - \cos\theta \lr{ \sin\theta \cos\phi, -\sin\theta\sin\phi, \cos\theta}
= \lr{ -\sin\theta \cos\theta \cos\phi, \sin\theta\cos\theta \sin\phi, \sin^2\theta}
= -\sin\theta \lr{ \cos\theta \cos\phi, -\cos\theta \sin\phi, -\sin\theta}.
\end{dmath}
\end{subequations}

The far field component of the electric fields directed towards the reflection points are

\begin{subequations}
\label{eqn:advancedantennaProblemSet3Problem3:380}
\begin{dmath}\label{eqn:advancedantennaProblemSet3Problem3:400}
\BE_x
=  j \eta k \frac{ I_0 }{4 \pi r} e^{-j k r} \sin\theta
\lr{ \cos\theta\cos\phi, -\cos\theta \sin\phi, -\sin\theta}
\end{dmath}
\begin{dmath}\label{eqn:advancedantennaProblemSet3Problem3:420}
\BE_y
=  j \eta k \frac{ I_0 }{4 \pi r} e^{-j k r} \sin\theta
\lr{ \cos\theta \cos\phi, -\cos\theta \sin\phi, -\sin\theta}.
\end{dmath}
\end{subequations}

After reflection, for \( r \gg h \), the direction of each of these fields will be approximately \( -\thetacap \), with an inversion from a reflection coefficient \( R = -1 \) due to the horizontal dipole configuration.  The superposition calculation requires only the summing of the phases that differ due to optical path length differences.

All the pieces required to start answering the specific subquestions of this problem are now at hand.  These will be answered below slightly out of order, starting with

\makeSubAnswer{}{advancedantenna:problemSet3:3c}

The path length for the image sources are

\begin{dmath}\label{eqn:advancedantennaProblemSet3Problem3:440}
k r_x
= k \Abs{ r ( \sin\theta \cos\phi, \sin\theta \sin\phi, \cos\theta) -(-2h, 0,0)}
= k r \Abs{ ( \sin\theta \cos\phi + 2h/r, \sin\theta \sin\phi, \cos\theta)}
= k r \sqrt{ ( \sin\theta \cos\phi + 2h/r)^2 + \sin^2\theta \sin^2\phi + \cos^2\theta) }
= k r \sqrt{ 1 + 4 h^2/r^2 + 4 (h/r) \sin\theta \cos\phi }
\approx k r \sqrt{ 1 + 4 (h/r) \sin\theta \cos\phi }
\approx k r \lr{ 1 + 2 (h/r) \sin\theta \cos\phi }
= k r + 2 k h \sin\theta \cos\phi,
\end{dmath}

and

\begin{dmath}\label{eqn:advancedantennaProblemSet3Problem3:460}
k r_y
= k \Abs{ r ( \sin\theta \cos\phi, \sin\theta \sin\phi, \cos\theta) -(0, -2h,0)}
= k r \Abs{ ( \sin\theta \cos\phi, \sin\theta \sin\phi + 2h/r, \cos\theta)}
= \cdots
= k r + 2 k h \sin\theta \sin\phi
\end{dmath}

The superposition of the LOS and reflection (image source) fields, in the far field, is

%\begin{dmath}\label{eqn:advancedantennaProblemSet3Problem3:480}
\boxedEquation{eqn:advancedantennaProblemSet3Problem3:480}{
\BE
=  j \eta k \frac{ I_0 }{4 \pi r} e^{-j k r} \sin\theta
\lr{ 1 - e^{-j k h \sin\theta \cos\phi } - e^{-j k h \sin\theta \sin\phi } } \thetacap.
}
%\end{dmath}

This provides the array factor

%\begin{dmath}\label{eqn:advancedantennaProblemSet3Problem3:500}
\boxedEquation{eqn:advancedantennaProblemSet3Problem3:500}{
\textrm{AF}
= 1 - e^{-j k h \sin\theta \cos\phi } - e^{-j k h \sin\theta \sin\phi }.
}
%\end{dmath}

What will be of more interest isn't the array factor, but its absolute square

\begin{dmath}\label{eqn:advancedantennaProblemSet3Problem3:520}
\Abs{\textrm{AF}}^2
=
\lr{ 1 - e^{-j k h \sin\theta \cos\phi } - e^{-j k h \sin\theta \sin\phi } }
\lr{ 1 - e^{j k h \sin\theta \cos\phi } - e^{j k h \sin\theta \sin\phi } }
=
3 - 2 \cos\lr{k h \sin\theta \cos\phi } - 2 \cos\lr{k h \sin\theta \sin\phi }
+ 2 \cos\lr{k h \sin\theta (\cos\phi - \sin\phi) }.
\end{dmath}

\makeSubAnswer{}{advancedantenna:problemSet3:3a}

The radiation intensity is

\begin{equation}\label{eqn:advancedantennaProblemSet3Problem3:540}
U
= \inv{2} \eta \lr{ \frac{k I_0  }{4 \pi} }^2 \sin^2 \theta \Abs{\textrm{AF}}^2
= B_0 \sin^2 \theta
\Abs{\textrm{AF}}^2.
\end{equation}

This holds for both isolated antenna with \( \textrm{AF} = 1 \), and the corner cube with \( \Abs{\textrm{AF}}^2 \) given by \cref{eqn:advancedantennaProblemSet3Problem3:520}.

For the isolated antenna, the radiation intensity is maximized at \( \theta = \pi/2 \), so

\begin{dmath}\label{eqn:advancedantennaProblemSet3Problem3:560}
D_{0,\textrm{iso}}
= \frac{4 \pi \times 1}{2 \pi \int_0^\pi \sin^3\theta d\theta}
= \frac{2}{4/3}
= \frac{3}{2}.
\end{dmath}

For the corner cube the maximization problem is trickier.  As an approximation, if \( k h \) is assumed to be small, then all the cosines in \cref{eqn:advancedantennaProblemSet3Problem3:520} are close to unity, so again at \( \theta = \pi/2 \)

\begin{equation}\label{eqn:advancedantennaProblemSet3Problem3:580}
\max \sin^2\theta \Abs{\textrm{AF}}^2 \approx 1 \times \lr{3 - 2 - 2 + 2} = 1.
\end{equation}

Despite this being equal to the maximum radiation intensity of the isolated radiator, the radiated power will be reduced by the corner cube configuration, which should increase the directivity.

That radiated power is

\begin{equation}\label{eqn:advancedantennaProblemSet3Problem3:600}
P_{\textrm{rad}}
=
B_0
\int_0^{\pi/2} d\phi
\int_0^{\pi} d\theta \sin^3 \theta
\Abs{\textrm{AF}}^2
\end{equation}

A first order expansion \( \cos (k h \alpha) \approx 1 - (k h \alpha)^2/2 \) gives

\begin{dmath}\label{eqn:advancedantennaProblemSet3Problem3:620}
\Abs{\textrm{AF}}^2
\approx
1 + (k h)^2 \sin^2 \theta \lr{
\cos^2\phi + \sin^2 \phi - (\cos\phi - \sin\phi)^2
}
=
1 + 2 (k h)^2 \sin^2 \theta \cos\phi \sin\phi.
=
1 + (k h)^2 \sin^2 \theta \sin( 2 \phi ).
\end{dmath}

With

\begin{dmath}\label{eqn:advancedantennaProblemSet3Problem3:640}
\int_0^{\pi/2} \sin (2\phi) d\phi = 1,
\end{dmath}

and

\begin{dmath}\label{eqn:advancedantennaProblemSet3Problem3:660}
\int_0^{\pi} \sin^5 \theta d\theta = 16/15,
\end{dmath}

the radiated power is

\begin{dmath}\label{eqn:advancedantennaProblemSet3Problem3:680}
P_{\textrm{rad}} \approx B_0 \lr{ \frac{\pi}{2} \frac{4}{3} + (k h)^2 \frac{15}{16}}.
\end{dmath}

Provided that \( k h \ll \sqrt{(2 \pi/3)(16/15)} \approx 1.5 \), the approximate directivity of the corner cube is

\begin{equation}\label{eqn:advancedantennaProblemSet3Problem3:700}
D_{0,\textrm{ccube}} \approx \frac{4 \pi \times 1}{2 \pi/3} = 6,
\end{equation}

a factor of 4 greater than the directivity of the isolated radiator.

\makeSubAnswer{}{advancedantenna:problemSet3:3b}

The radiation resistance was defined implicitly by the relation

\begin{equation}\label{eqn:advancedantennaProblemSet3Problem3:720}
P_{\textrm{rad}} = \inv{2} \Abs{I_0}^2 R_r,
\end{equation}

so the ratio of radiation resistance will just be the ratio of the radiated powers

\begin{equation}\label{eqn:advancedantennaProblemSet3Problem3:740}
\frac{R_{r,\textrm{ccube}}}{R_{r,\textrm{iso}}}
=
\frac{P_{\textrm{rad},\textrm{ccube}}}{P_{\textrm{rad},\textrm{iso}}}
=
\frac{2 \pi/3}{8 \pi/3}
=
\inv{4}.
\end{equation}

\makeSubAnswer{}{advancedantenna:problemSet3:3d}

The x-y plane is found at \( \theta = \pi/2 \) where the absolute square array factor is

\begin{equation}\label{eqn:advancedantennaProblemSet3Problem3:760}
\Abs{\textrm{AF}}^2 = 3
- 2 \cos\lr{ k h \cos\phi }
- 2 \cos\lr{ k h \sin\phi }
+ 2 \cos\lr{ k h (\cos\phi - \sin\phi) }.
\end{equation}

With \( h = \alpha \lambda \), this is

\begin{equation}\label{eqn:advancedantennaProblemSet3Problem3:780}
\Abs{\textrm{AF}}^2 = 3
- 2 \cos\lr{ 2 \pi \alpha \cos\phi }
- 2 \cos\lr{ 2 \pi \alpha \sin\phi }
+ 2 \cos\lr{ 2 \pi \alpha (\cos\phi - \sin\phi) }.
\end{equation}

A plot against both \( \alpha, \phi \) is found in \cref{fig:arrayFactorXY:arrayFactorXYFig4}, which shows that there are generally four lobes for any value of \( h \).

\mathImageFigure{../figures/ece1229-antenna-figures/arrayFactorXYFig4}{Plot of \( \Abs{\textrm{AF}}^2 \) in XY plane with \( \alpha = h/\lambda \).}{fig:arrayFactorXY:arrayFactorXYFig4}{0.3}{ps3:ps3Q3plotsAndMiscIntegrals.nb}

This is also plotted in
\cref{fig:arrayFactorXY:arrayFactorXYFig5} for a few selected values of \( h = \alpha \lambda \).

\mathImageFigure{../figures/ece1229-antenna-figures/arrayFactorXYFig5}{Polar plot of \( \Abs{\textrm{AF}}^2 \) in XY plane for various values of \( \alpha = h/\lambda \).}{fig:arrayFactorXY:arrayFactorXYFig5}{0.3}{ps3:ps3Q3plotsAndMiscIntegrals.nb}

\makeSubAnswer{}{advancedantenna:problemSet3:3e}

The code for the numerical calculations can be found in \nbref{ps3:ps3Q3plotsAndMiscIntegrals.nb}.  The results are
%The listing of \cref{mat:advancedantennaProblemSet3Problem3:20} shows the code used for the numerical calculation.  The results are

\begin{equation}\label{eqn:advancedantennaProblemSet3Problem3:n}
\begin{aligned}
D_0[h = \lambda/8] &= 6.34939 \\
D_0[h = \lambda/4] &= 8.11917 \\
D_0[h = \lambda/2] &= 10.3841.
\end{aligned}
\end{equation}

}

%%%\everymath{}
%%%\begin{figure}
%%%\caption{Numeric directivity calculation}\label{mat:advancedantennaProblemSet3Problem3:20}
%%%\begin{shaded}
%%%\begin{mat}
%%%Uccube[a_, p_, t_] =
%%%  Sin[t]^2 (3 - 2 Cos[ 2 Pi a Sin[t] Cos[p] ] -
%%%     2 Cos[ 2 Pi a Sin[t] Sin[p]] +
%%%     2 Cos[ 2 Pi a Sin[t] (Cos[p] - Sin[p])] ) ;
%%%Umax[a_] :=
%%%  NMaximize[{Uccube[a, p, t], 0 <= p <= Pi/2,  0 <= t <= Pi}, {p, t}] ;
%%%Prad[a_] := NIntegrate[Uccube[a, p, t], {p, 0, Pi/2}, {t, 0, Pi}]  ;
%%%dir[a_] := 4 Pi Umax[a][[1]]/Prad[a]  ;
%%%\end{mat}
%%%\end{shaded}
%%%\end{figure}


