%
% Copyright � 2015 Peeter Joot.  All Rights Reserved.
% Licenced as described in the file LICENSE under the root directory of this GIT repository.
%
\makeoproblem{Patch antenna.}{advancedantenna:problemSet5:2}{2015 problem set 5, p2}{
\index{patch antenna}
\index{microstrip patch}

A microstrip patch antenna is printed on a substrate with \( h = 0.1588 \si{cm} \), \( \epsilon_r = 2.2 \) at \( f_0 = 10 \si{GHz} \).
Give your length answers in \si{cm}.

Using the transmission-line model :

\makesubproblem{}{advancedantenna:problemSet5:2a}
Calculate the width \( W \).

\makesubproblem{}{advancedantenna:problemSet5:2b}
Calculate the effective relative permittivity \( \epsilon_{\textrm{eff}} \).

\makesubproblem{}{advancedantenna:problemSet5:2c}
Calculate the length of the patch \( L_0 \)
if no fringing-field effects are accounted for.

\makesubproblem{}{advancedantenna:problemSet5:2d}
Calculate the corrected length \( L = L_0 - m \Delta L \)
where \( m = 2 \) and \( \Delta L \) is the correction due
to the fringing fields.

\makesubproblem{}{advancedantenna:problemSet5:2e}
Estimate the admittance of each radiating slot \( Y_\txts = G + j B \).

\makesubproblem{}{advancedantenna:problemSet5:2f}

Now transform \( Y_\txts \) of the second slot (right) to the plane of the first slot (left) using the
impedance transformation,

\begin{dmath}\label{eqn:advancedantennaProblemSet5Problem2:20}
Z_{\textrm{in} 2} = Z_0 \frac
{Z_\txts + j Z_0 \tan(\beta L) }
{Z_0 + j Z_\txts \tan(\beta L) }
\end{dmath}

where \( \beta = k_0 \sqrt{\epsilon_{\textrm{eff}}} \)
is the effective propagation constant, and use \( Z_0 = 26 \Omega \) as the
characteristic impedance of the microstrip line.
What is the value of \( Z_{\textrm{in} 2} \) and \( Y_{\textrm{in} 2} = 1/Z_{\textrm{in} 2} \).

\makesubproblem{}{advancedantenna:problemSet5:2g}
Based on the above, calculate the total input impedance of the patch antenna \( Z_{\textrm{in}} \)
at the terminals of the first slot.

\makesubproblem{}{advancedantenna:problemSet5:2h}
If the imaginary part of \( Z_{\textrm{in}} \)
is not zero, adjust the length parameter m (in \partref{advancedantenna:problemSet5:2d})
between \( 0 < 3 < m \) to make the patch resonant (i.e. make the imaginary part of \( Z_{\textrm{in}} \) vanish).
What is the new input impedance in this case?
} % makeoproblem

\makeanswer{advancedantenna:problemSet5:2}{
\makeSubAnswer{}{advancedantenna:problemSet5:2a}

\begin{dmath}\label{eqn:advancedantennaProblemSet5Problem2:40}
W
= \inv{2 f_0 \sqrt{\mu_0 \epsilon_0} } \sqrt{ \frac{2}{\epsilon_r + 1} }
= \frac{c}{2 f_0} \sqrt{ \frac{2}{2.2 + 1}}
= \frac{3 \times 10^8 \si{m/s} \times 100 \si{cm/m}}{(2) 10 \times 10^9 \si{s^{-1}}} \sqrt{ \frac{2}{2.2 + 1}}
= 1.186 \si{cm}.
\end{dmath}

\makeSubAnswer{}{advancedantenna:problemSet5:2b}

\begin{dmath}\label{eqn:advancedantennaProblemSet5Problem2:60}
\epsilon_{\textrm{eff}} =
\frac{\epsilon_\txtr + 1}{2}
+\frac{\epsilon_\txtr - 1}{2} \lr{ 1 + \frac{12 h}{W} }^{-1/2}
=
1.9716.
\end{dmath}

\makeSubAnswer{}{advancedantenna:problemSet5:2c}

\begin{dmath}\label{eqn:advancedantennaProblemSet5Problem2:80}
L_0
= \frac{\lambda_0}{2}
= \frac{c }{2 f_0 \sqrt{\epsilon_{\textrm{eff}}}}
= 1.068 \si{cm}.
\end{dmath}

\makeSubAnswer{}{advancedantenna:problemSet5:2d}

\begin{dmath}\label{eqn:advancedantennaProblemSet5Problem2:100}
\frac{\Delta L }{h}
= 0.412 \frac{\epsilon_{\textrm{eff}} + 0.3}{\epsilon_{\textrm{eff}} -0.258} \frac{\frac{W}{h} + 0.264}{\frac{W}{h} + 0.8}
= 0.5108.
\end{dmath}

\begin{dmath}\label{eqn:advancedantennaProblemSet5Problem2:120}
\Delta L = 0.5108 h
= 0.081 \si{cm}.
\end{dmath}

\begin{dmath}\label{eqn:advancedantennaProblemSet5Problem2:260}
L = L_0 - 2 \Delta L = 0.9061 \si{cm}
\end{dmath}

\makeSubAnswer{}{advancedantenna:problemSet5:2e}

\begin{dmath}\label{eqn:advancedantennaProblemSet5Problem2:300}
\lambda_0 = c/f_0 = 3 \si{cm}.
\end{dmath}

\begin{dmath}\label{eqn:advancedantennaProblemSet5Problem2:140}
k_0 = \frac{2 \pi}{\lambda_0} = 2.0944 \si{cm^{-1}}
\end{dmath}

The constraint for the calculation of \( G \) requires

\begin{dmath}\label{eqn:advancedantennaProblemSet5Problem2:320}
\frac{h}{\lambda_0} = 0.053 < \inv{10},
\end{dmath}

which is satisfied, so

\begin{dmath}\label{eqn:advancedantennaProblemSet5Problem2:160}
G
= \frac{W}{120 \lambda_0} \lr{ 1 - \inv{24} \lr{k_0 h}^2 }
= 0.0033 \mho.
\end{dmath}

\begin{dmath}\label{eqn:advancedantennaProblemSet5Problem2:180}
B
= \frac{W}{120 \lambda_0} \lr{ 1 - 0.636 \ln\lr{k_0 h} }
= 0.0056 \mho.
\end{dmath}

\begin{dmath}\label{eqn:advancedantennaProblemSet5Problem2:200}
Y_\txts = G + j B
= \lr{ 0.0033 + 0.0056 j } \mho
= 0.00649  \phase{\ang{60}} \mho.
\end{dmath}

\begin{dmath}\label{eqn:advancedantennaProblemSet5Problem2:210}
Z_\txts = \lr{ 78 - 133 j } \Omega = 154 \phase{-\ang{59}} \Omega.
\end{dmath}

\makeSubAnswer{}{advancedantenna:problemSet5:2f}

\begin{dmath}\label{eqn:advancedantennaProblemSet5Problem2:220}
Z_{\textrm{in} 2} = \lr{ 19 + 71 j } \Omega = 73 \phase{\ang{75}} \Omega.
\end{dmath}

\begin{dmath}\label{eqn:advancedantennaProblemSet5Problem2:240}
Y_{\textrm{in} 2} = \lr{ 0.0036 - 0.0131 j } \mho = 0.0136 \phase{ -\ang{75} } \mho.
\end{dmath}

\makeSubAnswer{}{advancedantenna:problemSet5:2g}

\begin{dmath}\label{eqn:advancedantennaProblemSet5Problem2:280}
Z_{\textrm{in}}
= \inv{Y_\txts + Y_{\textrm{in} 2}}
= \lr{ 66 + 73 j } \Omega
= 98 \phase{\ang{48}} \Omega.
\end{dmath}

\makeSubAnswer{}{advancedantenna:problemSet5:2h}

The imaginary part of \( Z_{\textrm{in}} \) is plotted in \cref{fig:imagZin:imagZinFig6}.

\mathImageFigure{../figures/ece1229-antenna-figures/imagZinFig6}{\(\Imag Z_{\textrm{in}} \) variation with \( m \).}{fig:imagZin:imagZinFig6}{0.3}{ps5:pII.m}

The zero is found at

\begin{dmath}\label{eqn:advancedantennaProblemSet5Problem2:340}
m = 1.22097,
\end{dmath}

at which point the new impedance is

\begin{dmath}\label{eqn:advancedantennaProblemSet5Problem2:n}
Z_{\textrm{in}}
= 152.3 \Omega.
\end{dmath}
}
