%
% Copyright � 2015 Peeter Joot.  All Rights Reserved.
% Licenced as described in the file LICENSE under the root directory of this GIT repository.
%
\makeoproblem{Radar cross section.}{advancedantenna:problemSet2:1}{2015 problem set 2, p1}{
\index{radar cross section}

Consider a flat rectangular metallic plate of physical area \( A_\txtp \) (\si{m^2}). The incident normal power density is \( W_\txti \) (\si{W/m^2}).
Now consider the plate as a receiving and re-transmitting antenna having an effective aperture equal to its physical area (the plate is electrically large). Based on this idea, show that approximately the radar scattering cross section (RCS) of the plate is given by

\begin{dmath}\label{eqn:advancedantennaProblemSet2Problem1:20}
\sigma = \frac{ 4 \pi A_\txtp^2}{\lambda^2},
\end{dmath}

as we have seen in class.
\index{radar cross section}
\index{RCS}
\index{power density}

} % makeoproblem

\makeanswer{advancedantenna:problemSet2:1}{

A few simplifying assumptions are required to show this result:

\begin{enumerate}
\item All the power received is re-radiated as if from a point source.
\item The plate is a perfectly efficient re-radiator.
\item The effective aperture \(A_{\textrm{eff}}\) is also the maximum effective aperture, so it (and the directivity) has no directional dependence.
\end{enumerate}

The scattering geometry for this problem is sketched in \cref{fig:scatteringOffPlane:scatteringOffPlaneFig1}.

\imageFigure{../figures/ece1229-antenna-figures/scatteringOffPlaneFig1}{Scattering off of plane surface.}{fig:scatteringOffPlane:scatteringOffPlaneFig1}{0.15}

First note that the definition of the radar cross section \( \sigma \) is

\begin{equation}\label{eqn:advancedantennaProblemSet2Problem1:40}
\sigma \equiv \lim_{R \rightarrow \infty} 4 \pi R^2 \frac{W_s}{W_i},
\end{equation}

where \( W_s \) is the scattered power density, \( W_i \) is the incident power density, and \( R \) is the distance from the scattering object to the measurement point.  Without the point source approximation for the re-radiation of the incident power, this quantity would depend on the orientation of the plate with respect to the observation.

With constant normal incident power density, the captured power is
\index{captured power}

\begin{equation}\label{eqn:advancedantennaProblemSet2Problem1:60}
P_c = W_i A_p,
\end{equation}

where \( A_p \) is the area of the plate.  Treating all the power as radiated as if from a point source, measured at distance \( R \) from the plate, and assuming a perfect radiator (i.e. \( G = D_0 \)), the
scattered power density at this point of observation is

\begin{equation}\label{eqn:advancedantennaProblemSet2Problem1:80}
W_s = \frac{P_c G}{4 \pi R^2} = \frac{P_c D_0}{4 \pi R^2}.
\end{equation}
\index{scattered power density}

The directivity follows from the assumption that the effective area equals the physical area, since that means

\begin{equation}\label{eqn:advancedantennaProblemSet2Problem1:100}
A_{\textrm{eff}} \equiv \frac{\lambda^2 D_0}{4 \pi} = A_p,
\end{equation}

so

\begin{equation}\label{eqn:advancedantennaProblemSet2Problem1:120}
D_0 = \frac{4 \pi A_p}{\lambda^2}.
\end{equation}

The scattered power density at the receiver is

\begin{dmath}\label{eqn:advancedantennaProblemSet2Problem1:140}
W_s
= \frac{P_c}{4 \pi R^2} D_0
%= \frac{P_c}{4 \pi R^2} \frac{4 \pi A_p}{\lambda^2}
= \frac{W_i A_p}{4 \pi R^2} \frac{4 \pi A_p}{\lambda^2}.
\end{dmath}

Plugging this into \cref{eqn:advancedantennaProblemSet2Problem1:40}, and dropping the limit that becomes irrelevant, gives

\begin{dmath}\label{eqn:advancedantennaProblemSet2Problem1:160}
\sigma = \cancel{4 \pi R^2}
 \frac{\cancel{W_i} A_p}{\cancel{4 \pi R^2}} \frac{4 \pi A_p}{\lambda^2} \inv{\cancel{W_i}}
= \frac{4 \pi A_p^2}{\lambda^2},
\end{dmath}

as desired.
}
